%% outline.tex
%% Jeremy Singer
%% 9 Oct2015

% outline proposal for EPSRC SADEA call

\documentclass[a4paper,11pt]{article}

%% preamble.tex
%% Jeremy Singer
%% 12 Dec 12

% page sizing hacks
\usepackage[left=1.8cm,top=1.8cm,right=1.8cm,bottom=1.75cm,nohead]{geometry}

% packages
\usepackage[breaklinks,colorlinks,citecolor=blue,urlcolor=blue,linkcolor=blue]{hyperref}
\usepackage{graphicx}
\usepackage{multicol}
\usepackage{url}
%%\usepackage{bibspacing}
\usepackage{color}
% section font
\usepackage[scaled]{helvet}
\usepackage{sectsty}
\allsectionsfont{\bfseries\sffamily}
% for paragraph spacing 
\usepackage[parfill]{parskip}
% for more compact lists
\usepackage{enumitem}
\setlist[itemize]{noitemsep}

%euro symbol
\usepackage{eurosym}

% fonts
\usepackage{paratype}
\renewcommand*\familydefault{\sfdefault} %% Only if the base font of the document is to be sans serif
\usepackage[T1]{fontenc}

%% serif font used for MarioNet proposal...
%%\usepackage{DejaVuSerifCondensed}
%%\usepackage[T1]{fontenc}

%\usepackage[sc]{mathpazo}
%\linespread{1.05} % Palatino needs more leading (space between lines)
%\usepackage[T1]{fontenc}


% spacing
% % Less space around subsections
% \renewcommand{\subsection}{\@startsection
%   {subsection}% name
%   {2}% level
%   {0mm}% indent
%   {-0.5\baselineskip}% beforeskip
%   {0.01\baselineskip}% afterskip
%   {\normalsize\bfseries}}% style
\usepackage{titlesec}
\titlespacing{\section}{0pt}{\parskip}{-\parskip}
\titlespacing{\subsection}{0pt}{\parskip}{-\parskip}
\titlespacing{\subsubsection}{0pt}{\parskip}{-\parskip}


\setlength{\parskip}{0.4em}
\linespread{0.9}

% counter
\newcounter{workpackage}
\setcounter{workpackage}{0}
\newcounter{deliverable}
\setcounter{deliverable}{0}

% LaTeX macro stolen from Richard Jones
% params are 1 - workpackage title, 
%            2 - label (for future \ref)
%            3 - description
%            4 - deliverable
%            5 - risk

\newcommand{\WP}[5]{%
\vspace{0.5\baselineskip}
\noindent\textbf{WP~\refstepcounter{workpackage}\label{#2}\theworkpackage, #1}: 
#3 
\underline{\emph{Deliverables}}:~#4  
\ifx&#5&%
  % #8 is empty
\else
  \noindent\underline{\emph{Risk}}:~#5\\ 
\fi
}

\newcommand{\anycomment}[2][red]{{\textcolor{#1}{\,[\emph {#2}\,]}}}
\newcommand{\fixme}[1]{\anycomment[red]{#1}}

% comments by named author
\newcommand{\commenttwo}[2]{\(\spadesuit\){\bf #1: }{\rm \sf #2}\(\spadesuit\)}
% Comment this out in the draft
%\renewcommand{\commenttwo}[2]{}
\newcommand{\pwtcomment}[1]{\commenttwo{PWT}{#1}}
\newcommand{\jscomment}[1]{\commenttwo{JS}{#1}}

% no page numbers
\pagestyle{empty}
\thispagestyle{empty}


\begin{document}

\title{\Large \bfseries Sustainable Datacenters \vspace{-4mm}}
\author{\vspace{-5mm} \textbf{\normalsize S.\ Cox, S.\ Johnston, C.\ Perkins, J.\ Singer,  P.\ Tso, D.\ White, E. Yoneki}}
\date{\vspace{-6mm}}
%\end{center}
\maketitle
\vspace{-5mm}
%
% short abstract/introduction needs to go here 
%
\section{Current Equipment}

one page, equipment to be linked.
Details of testbeds and teams. This is a track record page.

The beauty of the Raspberry Pi, and similar cut-down ARM development boards, is that they can be powered by AA batteries or miniature solar panels. Further, the compute units are cheap enough to be effectively disposable.

We plan to combine our resources and expertise to investigate distributed, sustainable, billion-core datacenters.
We want to investigate use-case scenarios for building federated 'virtual' datacenters from vast arrays of distrbuted, low-power, commodity processors.


\subsection{Cambridge: RaspiNet}

delay-tolerant networks
\cite{yoneki2014raspinet}

\subsection{Glasgow: Raspberry Pi Cloud}

The Glasgow Raspberry Pi Cloud has around 100 nodes spread between Glasgow and Liverpool. It is a fully featured cloud environment, running a software stack based on Linux and Docker. It is orchestrated by Kubernetes, which is standard cloud management software deployed by Google. 

scaled-down datacenters
\cite{tso2013glasgow}
- also at Liverpool JMU?


\subsection{Southampton: Iridis-Pi}

micro-supercomputers
\cite{cox2014iridis}

%%%%%%%%%%%%%%

\section{Added Value}

Why do we get more from inter-connecting our separate micro-testbeds?
Nationwide geo-distribution - what is the distance from Glasgow to Southampton? 400 miles?

We intend to vary the connectivity between the distributed micro-datacentres. We will range from data-mule style delay-tolerant networking (as currently used in Cambridge with the Iridium Satellite network) to ultra-high-speed interconnect via dedicated Janet LightPath routes, effectively creating a single logical datacenter from our disparate units. 
This controlled variation in connectivity will be part of our experimental framework, investigating the tradeoffs involved in the various connection possibilities.

leverage existing hardware and expertise at four UK sites
In our experience, the biggest challenge has been setting up software stacks on the RPi nodes. Now this is completed, we can work on compelling and innovative use-case scenarios.





\section{Strategic Case}

Importance of ARM in datacenter.

We want to develop compelling demonstrator scenarios. Examples might be:
- mining and recognition in disaster zones - imagine piloting drones to help rescuers locate victims of earthquakes via a decentralized network.
- rural datacenters - each local datacenter computes info locally, makes this available to local residents, in a dashboard format. Also uploads info to central repo when possible.
- a federation of personal micro-datacenters. Imagine using all the devices you have (smartphone, smartwatch, laptop) as a personal-area datacenter. Compute so as to conserve power in all - send appropriate jobs to appropriate places, etc. This leads onto the concept of peer-to-peer social networking. Abandon Facebook and let everyone have their own mailserver, blog and calendar. The management of this distributed infrastructure is a key research challenge.
- content-centric networking - caches everywhere. every router becomes a cache. Smart cache in the home - programmable infrastructure needed here.
- rural datacenter alliance. Renewable power drives small, self-sustaining micro-datacenters that federate to provide services, as opposed to the current paradigm of `big-blob' infrastructure.

We note that HiPEAC / EU / ARM are engaged in building a low-power supercomputer (Mont Blanc project). Our distinct position is that low-power processors should be deployed in other large-scale systems areas than `just' traditional high-performance computing.



\section{Consortium Development}

Initial stages---

1. harness our network

2. 
work within existing academic pools e.g.\ SICSA (Scottish universities) next-generation internet community, complex systems community (weblinks).
N8 - northern universities research partnership (Does Posco have links here?)
Science and Engineering South \url{http://www.ses.ac.uk} - Southampton and Cambridge are partners. This consortium already has equipment sharing agreements.

Then - expand.
3. with a community workshop, a public website, a timesharing facility.

4. volunteer involvement. Invite general public (public engagement and impact!) to contribute their Raspberry Pi resources to a distributed datacenter facility, then federate with us. Software packages and instructions will be readily available. Need some incentives here? Not sure what...



\section{National Importance}

ARM in the datacenter.

Distributed datacenters - avoid `data drain' of all our personal and corporate info ending up in US datacenters... cf.\ Cambridge data box idea. (Anil et al).

\section{Pathway to Impact / Sustainability}

Academic dissemination via publications, community workshop.

Public engagement via blog (existing at \url{http://raspberrypicloud.wordpress.com} - need to upgrade?).


\section{Resources and Management}

Aim for \pounds 1m max.
Say a four year project. Would this fund 1 grade 6 RA per site?

Technical support?

Also need money for equipment... although not too much since we should say we have most of it already!




\textbf{The total RC budget is \pounds 1.00m over 48 months} (FEC \pounds 1.250m) as shown below.
\\
\textbf{Staff:} 
\\
%\textbf{Total Staff: \pounds 1,116k} (FEC \pounds 1,395k)\vspace{1mm}
\\
\textbf{Travel:}
We request support for semi-annual all-hands project meetings,
cross-site researcher mobility,
attendance at 2x international conference per site per year.
Also money to host our own workshops.
\\
\textbf{Total Travel: \pounds 36k} (FEC \pounds 45k)\vspace{1mm}
\\
\textbf{Equipment:}
\\
\textbf{Total Equipment: \pounds 24k} (FEC \pounds 30k)
\\


%% \section{Related Work}
%% @jsinger - for outline call,
%% leave related work as references
%% inlined in other sections

{\small
\bibliographystyle{abbrv}
\bibliography{outline}
}

\end{document}

